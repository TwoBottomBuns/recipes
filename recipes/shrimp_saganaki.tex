\section{Shrimp Saganaki}

\begin{itemize}
\item 3 shallots finely chopped
\item 4 garlic cloves, roughly chopped
\item 28 oz can diced tomatoes
\item 1 teaspoon ground cumin
\item 0.5 teaspoon crushed red pepper flakes (use less if you are heat-sensitive)
\item 1 tablespoon honey
\item 1.5 pounds extra large shrimp (26/30), peeled and deveined, thawed if frozen
\item 6 ounces feta cheese
\item 0.75 teaspoon dried oregano
\item 2 tablespoons roughly chopped fresh mint
\end{itemize}

\begin{enumerate}
\item Preheat oven to 400, set one oven rack in the middle and the other about 5 inches underneath the broiler.
\item Cook the shallots and garlic over medium-low heat, stirring occasionally until softened (5-7 min- utes). Do not
    brown.
\item Add the following, bring to a boil, and reduce heat to medium-low and cook uncovered, stirring occasionally, until
    the sauce is thickened (15-20 minutes).
    \item tomatoes with their juices
    \item salt
    \item pepper
    \item cumin
    \item red pepper flakes
    \item honey
\item Transfer to a baking dish. Arrange the shrimp over the tomato sauce in an even layer. Crumble feta over the
    shrimp, and then sprinkle with oregano.
\item Bake until the shrimp are pink and just cooked (12-15 minutes).
\item Turn on the broiler and transfer the dish to the top oven rack, broiling until the feta is golden brown in spots
    (1-2 minutes).
\item Let the shrimp rest for 5 minutes, sprinkle with mint, and serve.
\end{enumerate}

\subsection{Note}

I like to make this with cheesy garlic bread to scoop up the saganki. Save some of the shallots from the recipe, and add
those to a pan with some garlic, oil, and butter. Cook until the butter is melted, try not to brown the garlic. Top
bread with the mixture, then cover with mozzarella cheese and bake along with the shrimp.
